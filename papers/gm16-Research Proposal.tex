\documentclass[12pt]{article}
\usepackage{amsmath}
\usepackage{titling}
\usepackage{siunitx}
\usepackage{pgfplots}
\usepackage{pgfplotstable}
\usepackage[a4paper, total={6.5in, 9in}]{geometry}

\setlength{\droptitle}{-3cm}

\begin{document}

\title{\textbf{Generating Educational Storybook Scenes By Deriving Semantic Meaning From Plaintext Using Machine Learning}}
\author{Gautam Mittal\\\\
	Advanced Authentic Research (AAR)\\
        Henry M. Gunn High School\\
		\texttt{gautam@mittal.net}}
\date{December 7, 2015}
\maketitle

% Abstract goes here
\textbf{Abstract.} Throughout history, technology has made a significant impact both inside and outside the classroom. Since the rise of commercially available computers, the ability to access the Internet and other computing resources has empowered educators and students alike. Educational platforms and applications have enriched the quality and delivery of education, especially for the younger generations of students. Early on in a student?s formal education, the first challenge they must overcome is learning to read. Part of this challenge includes conceptualizing the meaning of written words and attaching a visual association with text. However, breakthroughs in modern computing can help students overcome this challenge. Natural language processing (NLP), or the use of algorithms to glean semantic meaning from human-generated language and then present it in more natural ways, has the ability to help young students conceptualize the text they are learning to read.
% End abstract

\section{Research Question}
How can machine learning techniques be utilized to construct educational storybook scenes from plaintext to allow those learning to read to better conceptualize their literature?

\section{Background and Significance}
Throughout history, technology has made a significant impact both inside and outside the classroom. Since the rise of commercially available computers, the ability to access the Internet and other computing resources has empowered educators and students alike. Educational platforms and applications have enriched the quality and delivery of education, especially for the younger generations of students. Early on in a student?s formal education, the first challenge they must overcome is learning to read. Part of this challenge includes conceptualizing the meaning of written words and attaching a visual association with text. However, breakthroughs in modern computing can help students overcome these challenges.

Machine learning, one of the most quickly evolving and controversial fields of computing, can aid in solving the challenges that students face when learning to read. More specifically, natural language processing (NLP), or the use of algorithms to glean semantic meaning from human-generated language, has the ability to help young students conceptualize the text they are learning to read. In Peter Norvig and Stuart Russell?s Artificial Intelligence: A Modern Approach they characterize natural language to be ambiguous, which is part of the reason why it is so difficult for both humans and machines to perceive a singular meaning from a piece of text. NLP enables computers to classify and evaluate text through the use of mathematical probabilistic models that determine the significance of words. Once a machine has a basic understanding of the meaning of the text, other techniques such as visual processing can be used to help the computer paint a graphical interpretation of the text.

In the past, researchers have used NLP models to build search engines, automatically classify spam or non-spam email messages, and on the education front, automatically generate practice exercises and produce concise summaries. The study would build upon the widely unexplored educational applications of NLP, as well as use the fundamental algorithms that allow machines to efficiently classify and extract semantic meaning from text to generate a visual interpretation that can be used to allow young readers to conceptualize reading material through the help of these computer-generated animated scenes.

\section{Research Methodology}
The study will be broken up into several coherent segments. The first step in completing the research will be to design and train a natural language processing model to analyze and deriving semantics from plaintext. The designing and training of this model should be completed no later than February 1, 2016. The construction of the server-side engine that will utilize the natural language processing model to analyze plaintext will be completed alongside researching methods of visual processing algorithms and frameworks that can generate images based on results produced by the model. This step of the process will be completed by March 14, 2016. The final implementation for the research will be through the creation of an online service where users can generate a fully illustrated educational storybook by providing their own text. This will utilize both the visual processing and natural language processing models through a server-side engine that will be deployed on a website. This will be completed by March 31, 2016.


\end{document}
